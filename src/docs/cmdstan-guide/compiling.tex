\chapter{Compiling \CmdStan Executables}\label{compiling.chapter}

\noindent
Preparing a Stan program to be run involves two steps,
%
\begin{enumerate}
\item translating the Stan program to \Cpp, and
\item compiling the resulting \Cpp to an executable.
\end{enumerate}
%
This chapter discusses both steps, as well as their encapsulation into
a single \code{make} target.

\section{Translating and Compiling through {\tt\bfseries make}}\label{make-models.section}

The simplest way to compile a \CmdStan program is through the
\code{make} build tool, which encapsulates the translation and
compilation step into a single command.  The commands making up the
\code{make} target for compiling a model are described in the
following sections, and the following chapter describes how to run a
compiled model.

Before compiling a \CmdStan program, change directories to
\code{<cmdstan-home>}.

\subsection{Translating and Compiling Test Models}

There are a number of example models distributed with \CmdStan which
unpack into the path \code{examples}.  To build the simple example
\nolinkurl{examples/bernoulli/bernoulli.stan}, the following call to
\code{make} suffices.

%
The following call will build an executable form of the Bernoulli
estimator. On Windows, replace \code{bernoulli} with \code{bernoulli.exe}.
%
\begin{quote}
\begin{Verbatim}[fontshape=sl]
> make examples/bernoulli/bernoulli
\end{Verbatim}
\end{quote}
%
This will translate the model \code{bernoulli.stan} to a \Cpp file,
\code{bernoulli.hpp}, and compile a \CmdStan program using the
generated \Cpp file, putting the executable in
\nolinkurl{examples/bernoulli/bernoulli(.exe)}.

Stan programs do not need to be in the \code{<cmdstan-home>}
directory. The current limitation is that the target executable name
can not have spaces -- this includes the path to the
executable. Spaces in the full path can be avoided by using relative
paths. For Windows users, if using the full path, include the drive
letter and use forward slashes,
e.g. \verb+make c:/cmdstan/bernoulli/bernoulli.exe+.

\subsection{Dependencies in {\tt\bfseries make}}

When executing a \code{make} target, all its dependencies are checked
to see if they are up to date, and if they are not, they are rebuilt.
If the \code{make} target to build the Bernoulli estimator is invoked
a second time, it will see that it is up to date, and will not
recompile the program.

If the file containing the Stan program is updated, the next call to
\code{make} will rebuild the \CmdStan executable.


\subsection{Getting Help from the {\tt makefile}}

\CmdStan's \code{makefile}, which contains the top-level instructions to
\code{make}, provides extensive help in terms of targets and options. Invoke
\code{make} with the target \code{help}:
%
\begin{quote}
\begin{Verbatim}[fontshape=sl]
> make help
\end{Verbatim}
\end{quote}

\subsection{Options to \code{make}}

\CmdStan allows users to change compilers, library versions for Boost,
Eigen, and CVODES, as well as compilation options such as
optimization.

For a full list of options, see \refappendix{make-options}



\subsubsection{Clean Targets}

A very useful target is \code{clean-all}, invoked as
%
\begin{quote}
\begin{Verbatim}[fontshape=sl]
> make clean-all
\end{Verbatim}
\end{quote}
%
This removes the \CmdStan tools. This step is necessary when changing
compilers or other \code{make} options.
