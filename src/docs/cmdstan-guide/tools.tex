\chapter{Overview}

\noindent
\CmdStan is the command-line interface for Stan. The next two
chapters describe tools that are built as part of \CmdStan
installation: \code{stanc} and \code{stansummary}. The process of building a
\CmdStan executable from a Stan program is as follows:
%
\begin{enumerate}
  \item A Stan program is written to file with a \code{.stan}
    extension.
  \item \code{stanc} is used to translate the Stan program into a
    \Cpp file. This \Cpp file is not a full program that can be
    compiled to executable directly, but a translation from the Stan
    language into a \Cpp concept. Each interface will generate identical
    \Cpp for the same Stan program.
  \item A \CmdStan executable is generated from the \CmdStan source
    and the generated \Cpp. Each Stan program will have its own
    \CmdStan executable. The options to the \CmdStan executable are
    described in \refchapter{stan-cmd}.
\end{enumerate}

\section{GNU Make utility}

\CmdStan uses GNU Make utility to build CmdStan binaries and Stan programs
The CmdStan `makefile` defines named targets to build the former and
general rules to build the latter.
A call to {\tt make} has the following general syntax:
%
\begin{quote}
\begin{Verbatim}[fontshape=sl]
make <flags> <variable=value> <name>
\end{Verbatim}
\end{quote}
%
When invoked without any arguments at all, {\tt make} prints a help message
%
\begin{quote}
\begin{Verbatim}[fontshape=sl, fontsize=\footnotesize]
--------------------------------------------------------------------------------
CmdStan v2.23.0 help

  Build CmdStan utilities:
    > make build

    This target will:
    1. Install the Stan compiler bin/stanc from stanc3 binaries and build bin/stanc2.
    2. Build the print utility bin/print (deprecated; will be removed in v3.0)
    3. Build the stansummary utility bin/stansummary
    4. Build the diagnose utility bin/diagnose
    5. Build all libraries and object files compile and link an executable Stan program

    Note: to build using multiple cores, use the -j option to make, e.g., 
    for 4 cores:
    > make build -j4


  Build a Stan program:

    Given a Stan program at foo/bar.stan, build an executable by typing:
    > make foo/bar

    This target will:
    1. Install the Stan compiler (bin/stanc), as needed.
    2. Use the Stan compiler to generate C++ code, foo/bar.hpp.
    3. Compile the C++ code using cc . to generate foo/bar

  Additional make options:
    STANCFLAGS: defaults to "". These are extra options passed to bin/stanc
      when generating C++ code. If you want to allow undefined functions in the
      Stan program, either add this to make/local or the command line:
          STANCFLAGS = --allow_undefined
    USER_HEADER: when STANCFLAGS has --allow_undefined, this is the name of the
      header file that is included. This defaults to "user_header.hpp" in the
      directory of the Stan program.
    STANC2: When set, use bin/stanc2 to generate C++ code.

  Example - bernoulli model: examples/bernoulli/bernoulli.stan

    1. Build the model:
       > make examples/bernoulli/bernoulli
    2. Run the model:
       > examples/bernoulli/bernoulli sample data file=examples/bernoulli/bernoulli.data.R
    3. Look at the samples:
       > bin/stansummary output.csv


  Clean CmdStan:

    Remove the built CmdStan tools:
    > make clean-all

--------------------------------------------------------------------------------
\end{Verbatim}
\end{quote}
%





\section{Building the \CmdStan Tools}\label{build.section}

The easy way to build \CmdStan is through the use of make. From a
command line window, type:
%
\begin{quote}
\begin{Verbatim}[fontshape=sl,fontsize=\small]
> cd <cmdstan-home>
> make build
\end{Verbatim}
\end{quote}
%
This will build utilities \code{stansummary} and \code{diagnose} and instantiate the \code{stanc} binary.
If your computer has multiple cores and sufficient ram, the build process can be parallelized
by providing the \code{-j} option. For example, to build on 4 cores, type:
%
\begin{quote}
\begin{Verbatim}[fontshape=sl,fontsize=\small]
> cd <cmdstan-home>
> make -j4 build
\end{Verbatim}
\end{quote}
%
\textbf{Warning:} \ The \code{make} program may take 10+ minutes and
consume 2+ GB of memory to build \CmdStan.
