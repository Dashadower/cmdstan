
\chapter{Print Command for Output Analysis}\label{print-command.chapter}

\noindent
Stan is distributed with a print command that is able to read in the
output of one or more Markov chains and summarize the posterior fits.
This operation mimics the \code{print(fit)} command in RStan, which
itself was modeled on the print functions from R2WinBUGS and R2jags.

\section{Building the Print Command}

Stan's \code{print} command is built along with \code{stanc} into the
\code{bin} directory.  It can be compiled directly using the makefile
as follows from the home directory into which Stan was unpacked (here
written as \code{<stan-home>}).
%
\begin{quote}
\begin{Verbatim}
> cd <stan-home>
> make bin/print
\end{Verbatim}
\end{quote}
%
All the usual compiler options from Stan's makefile apply, such as
\code{O=\farg{N}} to set optimization level to \farg{N}, and
\code{CC=clang++} to set the compilation to use clang. 

\section{Running the Print Command}

The print command is executed on one or more samples.csv files.  These
files may be provided as command-line arguments separated by spaces.
That means that wildcards may be used, as they will be replaced by
space-separated file names by the operating system's command-line
interpreter. 

Suppose there are three samples files in a directory generated by
fitting a negative binomial model to a small data set.
%
\begin{quote}
\begin{Verbatim}[fontshape=sl]
> ls samples*.csv
\end{Verbatim}
%
\begin{Verbatim}
samples1.csv	samples2.csv	samples3.csv
\end{Verbatim}
%
\begin{Verbatim}[fontshape=sl]
> bin/print samples*.csv
\end{Verbatim}
\end{quote}
%
The result of \code{bin/print} is displayed in
\reffigure{bin-print-eg}.%
%
\footnote{RStan's and PyStan's output analysis print may be different
  than that in the command-line version of Stan.}
%
\begin{figure}
\begin{Verbatim}[fontsize=\footnotesize]
   Inference for Stan model: negative_binomial_model
   1 chains: each with iter=(1000); warmup=(0); thin=(1); 1000 iterations saved.

   Warmup took (0.054) seconds, 0.054 seconds total
   Sampling took (0.059) seconds, 0.059 seconds total

                   Mean     MCSE   StdDev    5%   50%   95%  N_Eff  N_Eff/s  R_hat
   lp__             -14  6.2e-02  1.0e+00   -16   -14   -13    283     4773   1.00
   accept_stat__   0.88  5.6e-03  1.8e-01  0.51  0.95   1.0   1000    16881   1.00
   stepsize__      0.30  1.3e-15  8.9e-16  0.30  0.30  0.30   0.50      8.5   1.00
   treedepth__      1.4  2.6e-02  8.0e-01  0.00   1.0   2.0    946    15978   1.00
   n_divergent__    1.4  0.0e+00  0.0e+00  0.00   0.0   0.0   1000    16949   1.00
   alpha             17  1.8e+00  2.5e+01   1.9   9.5    50    181     3054   1.00
   beta              10  1.1e+00  1.4e+01   1.2   6.2    31    181     3057   1.00

   Samples were drawn using hmc with nuts.
   For each parameter, N_Eff is a crude measure of effective sample size,
   and R_hat is the potential scale reduction factor on split chains (at 
   convergence, R_hat=1).
\end{Verbatim}
\vspace*{-6pt}
\caption{\small\it Example output from \code{bin/print}.  The model
  parameters are \code{alpha} and \code{beta}.  The values for each
  quantity are the posterior means, standard deviations, and
  quantiles, along with Monte-Carlo standard error, effective sample
  size estimates (per second), and convergence diagnostic statistic.
  These values are all estimated from samples. In addition to the
  parameters, the value \code{lp\_\_} is the total log probability
  computed by the model (up to an additive constant).  The quantity
  \code{accept\_stat\_\_} is the NUTS acceptance statistic used by
  NUTS for slice and Metropolis rejection, \code{stepsize\_\_} the
  step size used by NUTS in its Hamiltonian simulation, and
  \code{treedepth\_\_} is the depth of tree used by NUTS, which is the
  log (base 2) of the number of leapfrog steps taken during the
  Hamiltonian simulation.  \code{n\_divergent\_\_} gives the number
  of leapfrog iterations with diverging error; because NUTS terminates
  at the first divergent iteration this should always be either 0 or 1.}
\label{bin-print-eg.figure}
\end{figure}
%\end{quote}
%
The posterior is skewed to the high side, resulting in posterior means
($\alpha=17$ and $\beta=10$) that are a long way away from the posterior
medians ($\alpha=9.5$ and $\beta=6.2$);  the posterior median is the
value listed under \code{50\%}, which is the 50th percentile of the
posterior values.

For Windows, the forward slash in paths need to be converted to backslashes.


\subsection{Output of Print Command}

\subsubsection{\code{n\_divergent}}

Stan uses a symplectic integrator to approximate the exact solution of the Hamiltonian dynamics and when the step size is too large relative to the curvature of the log posterior this approximation can diverge and threaten the validity of the sampler. \code{n\_divergent} counts the number of iterations within a given sample that have diverged and any non-zero value suggests that the samples may be biased in which case the step size needs to be decreased. Note that, because sampling is immediately terminated once a divergence is encountered, \code{n\_divergent} should be only 0 or 1.


\section{Command-line Options}

In addition to the filenames, \code{print} includes three flags to customize the output.  

\begin{description}
\longcmd{help}
{Prints usage information}
{No help output by default}
%
\cmdarg{sig\_figs}{int}
{Sets the number of significant figures displayed in the output}
{Valid values: 0 \textless sig\_figs}
{default = \code{2} }
%
\cmdarg{autocorr}{int}
{Calculates and then displays the autocorrelation of the specified chain}
{Valid values: Any integer matching a chain index}
{No autocorrelation output by default}
%
\end{description}
