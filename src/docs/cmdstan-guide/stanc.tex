\chapter{{\tt\bfseries stanc}: Translating Stan to C++}\label{stanc.chapter}

\section{Preparing the \stanc Compiler Binary}

Before the Stan compiler can be used, the binary \stanc must be created, either from
the pre-built binaries included with the release or downloaded from the nightly build.
This can be done using the makefile as follows. For Mac and Linux:
%
\begin{quote}
\begin{Verbatim}[fontshape=sl]
> make bin/stanc
\end{Verbatim}
\end{quote}
%
For Windows:
%
\begin{quote}
\begin{Verbatim}[fontshape=sl]
> make bin/stanc.exe
\end{Verbatim}
\end{quote}
%

\section{The \stanc Compiler}

The \stanc compiler converts Stan programs to \Cpp concepts. The
first stage of compilation involves parsing the text of the Stan
program.  If the parser is successful, the second stage of compilation
generates \Cpp code.  If the parser fails, it will provide an error
message indicating the location in the input where the failure
occurred and reason for the failure.

The following example illustrates a fully qualified call to \stanc
to build the simple Bernoulli model. 

For Linux and Mac:
%
\begin{quote}
\begin{Verbatim}[fontshape=sl]
> cd <cmdstan-home>
> bin/stanc --name=bernoulli --o=bernoulli.hpp \
  examples/bernoulli/bernoulli.stan 
\end{Verbatim}
\end{quote}
%
The backslash (\Verb|\|) is a continuation of the same line and can be
omitted if the command is on a single line.

For Windows:
%
\begin{quote}
\begin{Verbatim}[fontshape=sl]
> cd <cmdstan-home>
> bin\stanc.exe --name=bernoulli --o=bernoulli.hpp ^
    examples/bernoulli/bernoulli.stan 
\end{Verbatim}
\end{quote}
%
(The caret (\Verb|^|) is a line continuation on Windows.)

This call specifies the name of the model, here {\tt bernoulli}.
This will determine the name of the class implementing the model in
the \Cpp code.  Because this name is the name of a \Cpp class, it must
start with an alphabetic character (\code{a--z} or \code{A--Z}) and
contain only alphanumeric characters (\code{a--z}, \code{A--Z}, and
\code{0--9}) and underscores (\code{\_}) and should not conflict with
any \Cpp reserved keyword.  

The \Cpp code implementing the class is written to the file
\code{bernoulli.hpp} in the current directory.  The final argument,
\code{bernoulli.stan}, is the file from which to read the Stan
program.

\section{Command-Line Options for {\tt\bfseries stanc}}

The \stanc  program has the following command-line syntax:
%
\begin{quote}
\code{> stanc [options] {\slshape model\_file}}
\end{quote}
%
The argument \code{\slshape model\_file} is a path to a Stan model
file ending in suffix \code{.stan}.  The options are as follows.
%
\begin{description}
%
\item[\tt {-}-help] 
\mbox{ } \\ 
Displays the manual page for \stanc.  If this option is selected,
nothing else is done.
%
\item[\tt {-}-version]
\mbox{ } \\ 
Prints the version of \stanc.  This is useful for bug reporting
and asking for help on the mailing lists.
%
\item[\tt {-}-name={\slshape class\_name}]
\mbox{ } \\ 
Specify the name of the class used for the implementation of the
Stan model in the generated \Cpp code.  
\\[2pt]
Default: {\tt {\slshape class\_name = model\_file}\_model}
%
\item[\tt {-}-o={\slshape cpp\_file\_name}]
\mbox{ } \\ 
Specify the name of the file into which the generated \Cpp is written.
\\[2pt]
Default: {\tt {\slshape cpp\_file\_name} = {\slshape class\_name}.hpp}
%
\item[\tt {-}-{\slshape allow\_undefined}]
\mbox{ } \\ 
Do not throw a parser error if there is a function in the Stan program
that is declared but not defined in the functions block.
%
\end{description}

\section{Using External \Cpp Code}

The {\tt --allow\_undefined} flag can be passed to the call to \stanc,
which will allow undefined functions in the Stan language to be parsed
without an error. We can then include a definition of the function in
a \Cpp header file. We typically control these options with two {\tt
  make} variables: \Verb|STANCFLAGS| and \Verb|USER_HEADER|. See
\refappendix{make-options} for more details.

The \Cpp file will not compile unless there is a header file that
defines a function with the same name and signature in a namespace
that is formed by concatenating the {\tt class\_name} argument to
\stanc documented above to the string {\tt \_namespace}.

For more details about how to write \Cpp code using the Stan Math
Library, see \url{https://arxiv.org/abs/1509.07164}. As an example,
consider the following variant of the Bernoulli example
%
\begin{quote}
\begin{Verbatim}
functions {
  real make_odds(real theta);
}
data {
  int<lower=0> N;
  int<lower=0,upper=1> y[N];
}
parameters {
  real<lower=0,upper=1> theta;
}
model {
  theta ~ beta(1,1);  // uniform prior on interval 0,1
  y ~ bernoulli(theta);
}
generated quantities {
  real odds;
  odds = make_odds(theta);
}
\end{Verbatim}
\end{quote}
%
Here the {\tt make\_odds} function is declared but not defined,
which would ordinarily result in a parser error. However, if you
put \Verb|STANCFLAGS = --allow_undefined| into the
{\tt make/local} file or into the {\tt stanc} call, then the above 
Stan program will parse successfully but would not compile when you 
call
%
\begin{quote}
\begin{Verbatim}[fontshape=sl]
> make examples/bernoulli/bernoulli # on Windows add .exe 
\end{Verbatim}
\end{quote}
%
To compile successfully, you need to write a file such as {\tt
examples/bernoulli/make\_odds.hpp} with the following lines
%
\begin{quote}
\begin{Verbatim}
namespace bernoulli_model_namespace {

  template <typename T0__>
  inline
  typename boost::math::tools::promote_args<T0__>::type
  make_odds(const T0__& theta, std::ostream* pstream__) {
    return theta / (1 - theta);
  }

}
\end{Verbatim}
\end{quote}
%
Thus, the following {\tt make} invocation should work
%
\begin{quote}
\begin{Verbatim}[fontshape=sl]
> STANCFLAGS=--allow_undefined \
USER_HEADER=examples/bernoulli/make_odds.hpp \
make examples/bernoulli/bernoulli # on Windows add .exe 
\end{Verbatim}
\end{quote}
%
or you could put \Verb|STANCFLAGS| and \Verb|USER_HEADER|
into the {\tt make/local} file instead of specifying them
on the command-line.

If the function were more complicated and involved functions
in the Stan Math Library, then you would need to prefix the
function calls with {\tt stan::math::}. The {\tt pstream\_\_}
argument is mandatory in the signature but need not be used
if your function does not print any output. To see the 
necessary boilerplate look at the corresponding lines in the
generated \Cpp file.
