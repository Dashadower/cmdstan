\chapter{{\tt\bfseries stanc}: Translating Stan to C++}\label{stanc.chapter}


CmdStan  translates Stan programs to C++ using the Stan compiler program which
is included in the CmdStan release `bin` directory as program `stanc`.

As of release 2.22, the CmdStan Stan to C++ compiler is 
written in OCaml.
This compiler is called ``stanc3'' and has has its own repository
https://github.com/stan-dev/stanc3, from which pre-built binaries
for Linux, Mac, and Windows can be downloaded.

Prior to release 2.22, the Stan compiler program was compiled from C++ source code
that was part of the core Stan library.
This C++ compiler is still available as program `bin/stanc2`.
This compiler is not longer being maintained, i.e., existing bugs will not be fixed
and new functions and features are only available in the stanc3 compiler.
Its intended use is as a diagnostic tool and backup for the new stanc3 compiler and

\section{Preparing the \stanc Compiler Binary}

Before the Stan compiler can be used, the binary \stanc must be created.
This can be done using the makefile as follows. For Mac and Linux:
%
\begin{quote}
\begin{Verbatim}[fontshape=sl]
> make bin/stanc
\end{Verbatim}
\end{quote}
%
For Windows:
%
\begin{quote}
\begin{Verbatim}[fontshape=sl]
> make bin/stanc.exe
\end{Verbatim}
\end{quote}
%
To build the `bin/stanc2` program,  specify:
%
\begin{quote}
\begin{Verbatim}[fontshape=sl]
> make bin/stanc2
\end{Verbatim}
\end{quote}
%


\section{The \stanc Compiler}

The \stanc compiler converts Stan programs to \Cpp concepts.
If the parser fails, it will provide an error
message indicating the location in the input where the failure
occurred and reason for the failure.

The following example illustrates a fully qualified call to \stanc
to build the simple Bernoulli model. 

For Linux and Mac:
%
\begin{quote}
\begin{Verbatim}[fontshape=sl]
> cd <cmdstan-home>
> bin/stanc --o=bernoulli.hpp \
  examples/bernoulli/bernoulli.stan 
\end{Verbatim}
\end{quote}
%
The backslash (\Verb|\|) is a continuation of the same line and can be
omitted if the command is on a single line.

For Windows:
%
\begin{quote}
\begin{Verbatim}[fontshape=sl]
> cd <cmdstan-home>
> bin\stanc.exe --o=bernoulli.hpp ^
    examples/bernoulli/bernoulli.stan 
\end{Verbatim}
\end{quote}
%
(The caret (\Verb|^|) is a line continuation on Windows.)

This call specifies the name of the model, here {\tt bernoulli}.
This will determine the name of the class implementing the model in
the \Cpp code.  Because this name is the name of a \Cpp class, it must
start with an alphabetic character (\code{a--z} or \code{A--Z}) and
contain only alphanumeric characters (\code{a--z}, \code{A--Z}, and
\code{0--9}) and underscores (\code{\_}) and should not conflict with
any \Cpp reserved keyword.  

The \Cpp code implementing the class is written to the file
\code{bernoulli.hpp} in the current directory.  The final argument,
\code{bernoulli.stan}, is the file from which to read the Stan program.

In practice, \stanc is invoked indirectly, via the GNU Make utility,
which contains rules that compile a Stan program
to its corresponding executable.
To build the simple Bernoulli model via the makefile rules
%
\begin{quote}
\begin{Verbatim}[fontshape=sl]
> make examples/bernoulli/bernoulli
\end{Verbatim}
\end{quote}
%
The makefile rules first invoke the \stanc compiler to translate the Stan model to C++ ,
then compiles and links the C++ code to a binary executable.
The makefile variable {\tt STANCFLAGS } can be used to
to override the default arguments to \stanc, e.g.,

%
\begin{quote}
\begin{Verbatim}[fontshape=sl]
> make STANCFLAGS="--include_paths=~/foo" examples/bernoulli/bernoulli
\end{Verbatim}
\end{quote}
%

To use the stanc2 compiler instead of the stanc3 compiler,
set the make option `STANC2`:
%
\begin{quote}
\begin{Verbatim}[fontshape=sl]
> make STANC2=TRUE examples/bernoulli/bernoulli
\end{Verbatim}
\end{quote}
%


\section{Command-Line Options for {\tt\bfseries stanc3}}

The stanc2 compiler has the following command-line syntax:
%
\begin{quote}
\code{> stanc [options] {\slshape model\_file}}
\end{quote}
%
The argument \code{\slshape model\_file} is a path to a Stan model
file ending in suffix \code{.stan}.  The options are as follows.
%
\begin{description}
%
\item[\tt {-}-help] 
\mbox{ } \\ 
Displays the complete list of stanc3 options, then exits.
%
\item[\tt  --version]
\mbox{ } \\ 
 Display stanc version number
%
\item[\tt  --name={\slshape class\_name}]
\mbox{ } \\ 
Specify the name of the class used for the implementation of the
Stan model in the generated \Cpp code.  
%
\item[\tt --o={\slshape cpp\_file\_name}]
\mbox{} \\
Specify the name of the file into which the generated \Cpp is written.
%
\item[\tt  --allow\_undefined]
\mbox{ } \\ 
Do not throw a parser error if there is a function in the Stan program
that is declared but not defined in the functions block.
%
\item[\tt  --include\_paths={\slshape dir1,...dirN}]]
\mbox{ } \\ 
Takes a comma-separated list of directories that may contain a file in an \#include directive (default = "")
%
\item[\tt  --use-opencl]
\mbox{ } \\ 
If set, will use additional Stan OpenCL features enabled in the Stan-to-C++ compiler.
%
\item[\tt --auto-format]
\mbox{ } \\ 
 Pretty prints the program to the console.
%
\item[\tt  --print-canonical]
\mbox{ } \\ 
 Prints the canonicalized program to the console.
%
\item[\tt  --print-cpp]
\mbox{ } \\ 
If set, output the generated C++ Stan model class to stdout.
%
\item[\tt  --O ]
\mbox{ } \\ 
Allow the compiler to apply all optimizations to the Stan code.WARNING: This is currently an experimental feature!
%
\item[\tt  --warn-uninitialized]
\mbox{ } \\ 
Emit warnings about uninitialized variables to stderr. Currently an experimental feature.
\end{description}
%
The compiler also provides a number of debug options which are
primarily of interest to stanc3 developers; use the {\tt --help}  option
to see the full set.

\section{Command-Line Options for {\tt\bfseries stanc2}}

The stanc2 compiler has the following command-line syntax:
%
\begin{quote}
\code{> stanc2 [options] {\slshape model\_file}}
\end{quote}
%
The argument \code{\slshape model\_file} is a path to a Stan model
file ending in suffix \code{.stan}.  The options are as follows.
%
\begin{description}
%
\item[\tt {-}-help] 
\mbox{ } \\ 
Displays the complete list of stanc2 options, then exits.
%
\item[\tt {-}-version]
\mbox{ } \\ 
Prints the version of \stanc.  This is useful for bug reporting
and asking for help on the mailing lists.
%
\item[\tt {-}-name={\slshape class\_name}]
\mbox{ } \\ 
Specify the name of the class used for the implementation of the
Stan model in the generated \Cpp code.  
\\[2pt]
Default: {\tt {\slshape class\_name = model\_file}\_model}
%
\item[\tt {-}-o={\slshape cpp\_file\_name}]
  \mbox{ } \\ 
Specify the name of the file into which the generated \Cpp is written.
\\[2pt]
Default: {\tt {\slshape cpp\_file\_name} = {\slshape class\_name}.hpp}
%
\item[\tt {-}-{\slshape allow\_undefined}]
\mbox{ } \\ 
Do not throw a parser error if there is a function in the Stan program
that is declared but not defined in the functions block.
%
\item[\tt  --include\_paths={\slshape dir1,...dirN}]]
\mbox{ } \\ 
Takes a comma-separated list of directories that may contain a file in an \#include directive (default = "")
%
\end{description}

\section{Using External \Cpp Code}

The {\tt --allow\_undefined} flag can be passed to the call to \stanc,
which will allow undefined functions in the Stan language to be parsed
without an error. We can then include a definition of the function in
a \Cpp header file. We typically control these options with two {\tt
  make} variables: \Verb|STANCFLAGS| and \Verb|USER_HEADER|. See
\refappendix{make-options} for more details.

The \Cpp file will not compile unless there is a header file that
defines a function with the same name and signature in a namespace
that is formed by concatenating the {\tt class\_name} argument to
\stanc documented above to the string {\tt \_namespace}.

For more details about how to write \Cpp code using the Stan Math
Library, see \url{https://arxiv.org/abs/1509.07164}. As an example,
consider the following variant of the Bernoulli example
%
\begin{quote}
\begin{Verbatim}
functions {
  real make_odds(real theta);
}
data {
  int<lower=0> N;
  int<lower=0,upper=1> y[N];
}
parameters {
  real<lower=0,upper=1> theta;
}
model {
  theta ~ beta(1,1);  // uniform prior on interval 0,1
  y ~ bernoulli(theta);
}
generated quantities {
  real odds;
  odds = make_odds(theta);
}
\end{Verbatim}
\end{quote}
%
Here the {\tt make\_odds} function is declared but not defined,
which would ordinarily result in a parser error. However, if you
put \Verb|STANCFLAGS = --allow_undefined| into the
{\tt make/local} file or into the {\tt stanc} call, then the above 
Stan program will parse successfully but would not compile when you 
call
%
\begin{quote}
\begin{Verbatim}[fontshape=sl]
> make examples/bernoulli/bernoulli # on Windows add .exe 
\end{Verbatim}
\end{quote}
%
To compile successfully, you need to write a file such as {\tt
examples/bernoulli/make\_odds.hpp} with the following lines
%
\begin{quote}
\begin{Verbatim}
namespace bernoulli_model_namespace {

  template <typename T0__>
  inline
  typename boost::math::tools::promote_args<T0__>::type
  make_odds(const T0__& theta, std::ostream* pstream__) {
    return theta / (1 - theta);
  }

}
\end{Verbatim}
\end{quote}
%
Thus, the following {\tt make} invocation should work
%
\begin{quote}
\begin{Verbatim}[fontshape=sl]
> STANCFLAGS=--allow_undefined \
USER_HEADER=examples/bernoulli/make_odds.hpp \
make examples/bernoulli/bernoulli # on Windows add .exe 
\end{Verbatim}
\end{quote}
%
or you could put \Verb|STANCFLAGS| and \Verb|USER_HEADER|
into the {\tt make/local} file instead of specifying them
on the command-line.

If the function were more complicated and involved functions
in the Stan Math Library, then you would need to prefix the
function calls with {\tt stan::math::}. The {\tt pstream\_\_}
argument is mandatory in the signature but need not be used
if your function does not print any output. To see the 
necessary boilerplate look at the corresponding lines in the
generated \Cpp file.
