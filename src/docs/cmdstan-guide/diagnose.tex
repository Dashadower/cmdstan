\chapter{{\tt\bfseries diagnose}: Diagnosing Biased Hamiltonian Monte Carlo Inferences}\label{diagnose.chapter}

\noindent
\CmdStan is distributed with a utility that is able to read in and
analyze the output of one or more Markov chains to check for the
following potential problems:

\begin{itemize}
\item Transitions that hit the maximum treedepth
\item Divergent transitions
\item Low E-BFMI values
\item Low effective sample sizes
\item High $\hat{R}$ values
\end{itemize}

The meanings of several of these problems are discussed in
\url{http://mc-stan.org/misc/warnings.html#runtime-warnings}
and \url{https://arxiv.org/abs/1701.02434}.


\section{Building the {\tt\bfseries diagnose} Command}

\CmdStan's \code{diagnose} command is built along with \code{stanc} into
the \code{bin} directory. It can be compiled directly using the
makefile as follows.
%
\begin{quote}
\begin{Verbatim}[fontshape=sl]
> cd <cmdstan-home>
> make bin/diagnose
\end{Verbatim}
\end{quote}
%

\section{Running the {\tt\bfseries diagnose} Command}

The \code{diagnose} command is executed on one or more output files,
which are provided as command-line arguments separated by spaces.
If there are no apparent problems with the output files passed to
\code{bin/diagnose}, it outputs a message that all transitions
are within treedepth limit and that no divergent transitions were found.

It problems are detected, it outputs a summary of the problem along with
possible ways to mitigate it. As an example,  we use the
the ``eight schools'' model from Stan's example models and
its corresponding data.%
%
\footnote{The model and associated data files are here:
  \begin{itemize}
  \item
    \url{https://github.com/stan-dev/example-models/blob/master/misc/eight_schools/eight_schools.stan}
  \item
    \url{https://github.com/stan-dev/example-models/blob/master/misc/eight_schools/eight_schools.data.R}
  \end{itemize}
}
The model is run with four chains and the random seed 12345, leaving
the output files \code{eight\_schools1.csv}, \code{eight\_schools2.csv},
etc. The \code{diagnose} command is then run as follows:
\begin{quote}
\begin{Verbatim}[fontshape=sl]
> bin/diagnose eight_schools*.csv
\end{Verbatim}
\end{quote}
The result of \code{bin/diagnose} is displayed in
\reffigure{bin-diagnose-eg}, indicating two problems, one with
divergent transitions, and one indicating a low E-BFMI, and possible
ways to solve these problems. The first problem indicates that the
parameter \code{delta} of the sampling algorithm needs to be
increased. Since the contents of \code{eight\_schools1.csv} contains
the lines
\begin{quote}
\begin{Verbatim}
#     adapt
#       engaged = 1 (Default)
#       gamma = 0.050000000000000003 (Default)
#       delta = 0.80000000000000004 (Default)
\end{Verbatim}
\end{quote}
this suggests that \code{delta} should be increased beyond
0.8. Following section~\ref{detailed-command-arguments.section}, this
suggests that the model perhaps should be rerun as follows:
\begin{quote}
\begin{Verbatim}[fontshape=sl]
> for i in {1..4}
  do
    ./eight_schools sample adapt delta=0.9 \
       random seed=12345 id=$i data \
       file=eight_schools.data.R \
       output file=eight_schools$i.csv &
  done
\end{Verbatim}
\end{quote}


\begin{figure}
\begin{Verbatim}[fontsize=\footnotesize]
  95 of 4000 (2.4%) transitions ended with a divergence.  These
  divergent transitions indicate that HMC is not fully able to explore
  the posterior distribution.  Try rerunning with adapt delta set to a
  larger value and see if the divergences vanish.  If increasing adapt
  delta towards 1 does not remove the divergences then you will likely
  need to reparameterize your model.

  The E-BFMI, 0.27, is below the nominal threshold of 0.3 which
  suggests that HMC may have trouble exploring the target
  distribution.  You should consider any reparameterizations if
  possible.
\end{Verbatim}
\caption{Example output from \code{bin/diagnose}.}
\label{bin-diagnose-eg.figure}
\end{figure}

The online references (\url{http://mc-stan.org/misc/warnings.html#runtime-warnings}
and \url{https://arxiv.org/abs/1701.02434}) contain suggestions for
other diagnostic warning; however the correct resolution is
necessarily model specific, hence all suggestions general guidelines only.
