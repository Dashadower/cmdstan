\chapter{Overview}

\noindent
This document is a user's guide for the \CmdStan interface to the
Stan statistical modeling language. \CmdStan takes Stan programs
and generates executables that can be run directly from the command
line. \CmdStan is one of several interfaces to Stan; there are also R,
Python, Matlab, Julia, and Stata interfaces.

\section{Stan Home Page}\label{home-page.section}

For links to up-to-date code, examples, manuals, bug reports, feature
requests, and everything else Stan related, see the Stan home page:
%
\begin{quote}
\url{http://mc-stan.org/}
\end{quote}


\section{Licensing}

CmdStan, Stan, and the Stan Math Library are licensed under the new
BSD license (3-clause).  See \refappendix{licensing} for details,
including licensing terms for the dependent packages Boost, Eigen,
Sundials, and Intel TBB.


\section{Modeling Language User's Guide and Reference}

Stan's modeling language is shared across all of its interfaces.
Stan's language, along with a programming guide and many example
models, is detailed in the {\it Stan Modeling Language User's Guide
  and Reference Manual}, which is available from the Stan home page
(see \refsection{home-page}).

\section{Example Models}

There are many example models for Stan, in addition to those in the
user's guide and reference. These are all linked from the Stan home
page (see \refsection{home-page}).


\section{Benefits of \CmdStan}

Although \CmdStan has the least amount of functionality among the
Stan interfaces, the minimal nature of \CmdStan makes it trivial to
install and use the latest development version of the Stan
library. It also has the fewest dependencies, which makes it easier to
run in limited environments such as clusters. The output generated is
in CSV format and can be post-processed using other Stan interfaces or
general tools.
