\chapter{Installation and Compatibility}\label{install.appendix}

\noindent
This appendix describes the hardware and software required to run
\Stan.  The software includes \Stan and its libraries, as well as a
contemporary \Cpp compiler.  \Stan requires hardware powerful enough
to build and execute the models.  Ideally, that will be a 64-bit
computer with at least 4GB of memory and multiple processor cores.

\section{Operating System}

\Stan is written in portable \Cpp without {\Cpp}11 features, as are the
libraries on which it depends.  Therefore, \Stan should run on any machine
for which a suitable \Cpp compiler is available.  In practice, \Stan,
like the Boost and Eigen libraries on which it depends, is very hard
on the compiler and linker.

\Stan has been tested on the following operating systems.
%
\begin{itemize}
\item Linux (Debian, Ubuntu, Red Hat), 
\item Mac OS X (Snow Leopard, Lion, Mountain Lion), and
\item Windows (XP, 7, 8).
\end{itemize}
%
\Stan should work on other versions of these operating systems if
compatible \Cpp compilers can be found.  The plan is to keep up with
new versions of these operating systems and gradually phase out
testing on older versions.


\section{Step-by-Step Mac Install Instructions}

This section provides step-by-step install instructions for the Mac;
Linux and Windows sections follow.  It repeats the step-by-step
install instructions on Stan's home page at \url{http://mc-stan.org/}.

Stan has been tested on Mac OS X versions Snow Leopard, Lion, and
Mountain Lion.

\subsection{Tips for Mac Users}

\subsubsection{Finding and Opening Mac Applications and Files}

To open an application, use \code{[Command-Space]} (press both keys at
once on the keyboard) to open Spotlight, enter the application's name
in the text field, then click on the application in the pop-up menu or
\code{[Return]} if the right file or application is highlighted.

Spotlight can be used in the same way to find files or folders,
such as the default \code{Downloads} folder for web downloads.

\subsubsection{Open a Terminal for Shell Commands}

To run shell commands, open the built-in Terminal application (see the
previous subsection for details on how to find and open applications).

\subsection{Install Xcode C++ Development Environment}

The easiest (but not the only) way to install a \Cpp
development environment on a Mac is to use Apple's Xcode
development environment.

From the Xcode home page, 
%
\begin{quote}
\url{https://developer.apple.com/xcode/}
\end{quote}
%
click \code{View in Mac App Store}.

From the App Store, click \code{Install}, enter an Apple ID,
and wait for Xcode to finish installing.

Open the Xcode application, click top-level menu
\code{Preferences}, click top-row button \code{Downloads},
click button for \code{Components}, click on the \code{Install}
button to the right of the \code{Command Line Tools} entry, then
wait for it to finish installing.  

Click the top-level menu item \code{Xcode}, then click item 
\code{Quit Xcode} to quit.

To test, open the Terminal application and enter

\begin{quote}
\begin{Verbatim}[fontshape=sl,fontsize=\small]
> make --version
> g++ --version
\end{Verbatim}
\end{quote}
%
Verify that \code{make} is at version 3.81 or later and \code{g++}
is at 4.2.1 or later.


\subsection{Download and Unpack Stan Source}

Download the most recent version of \code{stan-2.m.p.tar.gz}
(\code{m} is the minor version and \code{p} the patch level) from
the Stan downloads list,
%
\begin{quote}
\url{https://github.com/stan-dev/stan/releases}
\end{quote}

Open the folder containing the download in the Finder
(typically, the user's top-level \code{Downloads} folder).

If the Mac OS has not automatically unpacked the \code{.tar.gz}
file into file \code{stan-2.m.p.tar},
double-click the \code{.tar.gz} file to unpack.

Double click on the \code{.tar} file to unarchive
directory \code{stan-2.m.p}.

Move the resulting directory to a location where it will not be
deleted, henceforth called \code{<stan-home>}.


\section{Step-by-Step Linux Install Instructions}

Stan has been tested on various Linux installations, including
Ubuntu, Debian, and Red Hat.

\subsection{Installing C++ Development Tools}

On Linux, \Cpp compilers and \code{make} are often installed by default.

To see if the \code{g++} compiler and \code{make} build system
are already installed, use the commands
%
\begin{quote}
\begin{Verbatim}[fontshape=sl,fontsize=\small]
> g++ --version
\end{Verbatim}
\end{quote}
% 
and
%
\begin{quote}
\begin{Verbatim}[fontshape=sl,fontsize=\small]
> make --version
\end{Verbatim}
\end{quote}
%

If these are at least at \code{g++} version 4.2.1 or later and \code{make}
version 3.81 or later, no additional installations are necessary.  It
may still be desirable to update the \Cpp compiler \code{g++}, because
later versions are faster.

To install the latest version of these
tools (or upgrade an older version), use the commands
%
\begin{quote}
\begin{Verbatim}[fontshape=sl,fontsize=\small]
> sudo apt-get install g++ 
\end{Verbatim}
\end{quote}
%
and
%
\begin{quote}
\begin{Verbatim}[fontshape=sl,fontsize=\small]
> sudo apt-get install make 
\end{Verbatim}
\end{quote}
% 
A password will likely be required by the superuser command \code{sudo}.


\subsection{Downloading and Unpacking Stan Source}

Download the most recent stable version of Stan,
\code{stan-2.m.p.tar.gz}, where \code{m} is the minor version and
\code{p} the patch level), from the Stan downloads page,
%
\begin{quote}
\url{https://github.com/stan-dev/stan/releases}
\end{quote}
%
to the directory where Stan will reside.

In a command shell, change directories to where the
tarball was downloaded, say \code{<download-dir>}, with
%
\begin{quote}
\begin{Verbatim}[fontshape=sl,fontsize=\small]
> cd <download-dir>
\end{Verbatim}
\end{quote}
%
where \code{<download-dir>} is replaced with the actual path to the directory.

Then, unpack the distribution into the subdirectory
\begin{quote}
\nolinkurl{<download-dir>/stan-2.m.p}
\end{quote}
%
with
%
\begin{quote}
\begin{Verbatim}[fontshape=sl,fontsize=\small]
> tar -xzf stan-2.m.p.tar.gz
\end{Verbatim}
\end{quote}



\section{Step-by-Step Windows Install Instructions}

Stan has been tested on Windows XP, Windows 7, and Windows 8.  

Stan also runs under Cygwin, which provides a unix-like shell on top
of Windows.  Instructions for Cygwin installation are provided below
in their own subsection.


\subsection{Windows Tips}

\subsubsection{Opening a Command Shell}

To open a Windows command shell, first open the \code{Start Menu}
(usually in the lower left of the screen), select option
\code{All Programs}, then option \code{Accessories}, then
program \code{Command Prompt}.

Alternatively, enter \code{[Windows+r]} (both keys together on the
keyboard), and enter \code{cmd} into the text field that pops up in
the Run window, then press \code{[Return]} on the keyboard to run.

\subsubsection{32-bit Builds}

\Stan defaults to a 64-bit build. On a 32-bit operating system, set
the \verb|BIT| variable to 32. For example, to build the Bernoulli
model in \refsection{compiling-model}, replace the original command
with:
%
\begin{quote}
\begin{Verbatim}[fontshape=sl]
> make BIT=32 src/models/basic_estimators/bernoulli
\end{Verbatim}
\end{quote}


\subsection{Rtools C++ Development Environment}

The simplest way to install a full \Cpp build environment that will
work for Stan is to use the Rtools package designed for R developers
on Windows (even if you don't plan to use R).

First, download the latest \emph{frozen} (i.e., stable) version of
Rtools from the Rtools home page, using
%
\begin{quote}
\url{http://cran.r-project.org/bin/windows/Rtools/}
\end{quote}

Next, double click on the downloaded file to open the Rtools
install wizard, then proceed through its options.
\begin{itemize}
\item \emph{Language}: select language, click \code{Next},
\item \emph{Welcome}: click \code{Next},
\item \emph{Information}: click \code{Next},
\item \emph{Setup Location}: accept default (\Verb|c:\Rtools|), click \code{Next},
\item \emph{Select Components}: select default, \code{Package
   Authoring}, click \code{Next},
\item \emph{Select Additional Tasks}: check \code{Edit Path} and \code{Save
 Version in Registry}, click \code{Next},
\item \emph{System Path Report}: ensure that that the paths to \code{c:\textbackslash{}Rtools\textbackslash{}bin} and \code{c:\textbackslash{}Rtools\textbackslash{}gcc-4.6.3\textbackslash{}bin} are listed at the beginning of the path and click \code{Next},
\item \emph{Ready to Install}: click \code{Next}, wait for the
  install to complete, then
\item \emph{Finish}: click \code{Finish}.
\item \emph{Confirm Path}: After the install has completed, open a command prompt and type \code{PATH} to ensure that the new path is activated and the Rtools folders are in the system path.
\end{itemize}

\subsection{Downloading and Unpacking Stan}

The Stan source code distributions are named
\code{stan-2.m.p.tar.gz}, where \code{m} is the minor version and
\code{p} the patch level.

Download the latest Stan source from the Stan downloads page,
%
\begin{quote}
\url{https://github.com/stan-dev/stan/releases}
\end{quote}
%
to any non-temporary folder.  (If in doubt, select \code{My Documents}
on Windows XP or \code{Documents} on Windows 7.)  

Change to the download directory (aka folder) using one of the
following commands, replacing \code{<username>} with
a Windows user name.
%
\begin{itemize}
\item \emph{Windows XP}: \ From the default starting directory, use
the following commands (quotes and all):
\begin{quote}
\begin{Verbatim}[fontshape=sl,fontsize=\small]
> cd "My Documents"
\end{Verbatim}
\end{quote}
%
The full path (including quotes) will work from anywhere,
\begin{quote}
\begin{Verbatim}[fontshape=sl,fontsize=\small]
> cd "c:\Documents and Settings\<username>\My Documents"
\end{Verbatim}
\end{quote}
\item \emph{Windows 7}:  \  From the default starting directory, use
\begin{quote}
\begin{Verbatim}[fontshape=sl,fontsize=\small]
> cd Documents
\end{Verbatim}
\end{quote} 
or use the full path, including quotes, from anywhere,
\begin{quote}
\begin{Verbatim}[fontshape=sl,fontsize=\small]
> cd "c:\Users\<username>\Documents"
\end{Verbatim}
\end{quote}
\end{itemize}
%
To verify that the downloaded Stan \code{.tar.gz} file is there,
list the directory contents using:
%
\begin{quote}
\begin{Verbatim}[fontshape=sl,fontsize=\small]
> dir
\end{Verbatim}
\end{quote}

Finally, unpack the distribution using the \code{tar} command (which is
installed as part of Rtools).
%
\begin{quote}
\begin{Verbatim}[fontshape=sl,fontsize=\small]
> tar --no-same-owner -xzf stan-2.m.p.tar.gz 
\end{Verbatim}
\end{quote}
%
The \code{--no-same-owner} flag is not strictly necessary,
but it removes a bunch of irrelevant warnings.


\subsection{64-bit Cygwin Install Instructions}

Stan can be run under Cygwin, the Unix look-and-feel environment for
Windows.  Cygwin must have recent versions of \code{make} and
\code{g++} (part of gcc) installed.  Within a Cygwin shell, Stan
will behave as under other Unixes.

Thanks to Kevin van Horn for mailing the following instructions into
the Stan-users mailing list.  They only cover 64-bit R and 64-bit
Cygwin, but that is what you should be using for Stan anyway.

\begin{enumerate}
\item Kill all Cygwin bash shells and shut down R.
\item After installing R and Rtools, make sure that R and Rtools are
  in the \code{PATH} environment variable. My R installation directory was
  \Verb|c:\Program Files\R\R-3.0.1| and my Rtools installation
  directory was \Verb|c:\Rtools|, so I added the following to the end
  of my user \code{PATH} variable:
  \begin{itemize}
  \item \Verb|C:\Program Files\R\R-3.0.1\bin\x64|
  \item \Verb|C:\Rtools\bin|
  \item \Verb|C:\Rtools\gcc-4.6.3\bin|
  \end{itemize}
\item Now there could be a conflict between Cygwin and Rtools when
  running a bash shell under Cygwin, so I added the following lines to
  my \Verb|.bash_profile| file to remove any \code{PATH} directory
  referencing Rtools:
  \begin{Verbatim}
> TMP=`echo $PATH | /usr/bin/tr ':' '\n'                      \
                  | /usr/bin/egrep -iv '^/cygdrive/c/Rtools/' \
                  | tr '\n' ':'`
> PATH=${TMP%:}
\end{Verbatim}
(Note that the backslash characters signal that the line continues
after a return.)  This was only necessary to allow me to continue
using Cygwin.
\item Apparently there is something in \code{Rcpp} or \code{inline} or
  \code{rstan} that doesn't like \code{UNC} paths. My home directory
  was \Verb|\\server\users\kevinv| and apparently this caused my
  local R library directory to be \Verb|\server\users\kevinv\R|
  as verified by \code{.libPaths()} from the R command prompt. 

  I fixed this by copying the entire directory tree rooted at
  \Verb|\server\users\kevinv\R| over to a local directory on my workstation,
  \Verb|C:\Users\KevinV\R|, then adding the user environment variable
  \Verb|R_LIBS_USER=C:\Users\KevinV\R|. I shut down R and restarted it.
\item At this point the instructions given for installing Rstan
  finally worked.
\end{enumerate}

\subsection{MKL Compiler Instructions}

\subsubsection{Getting the MKL}

To purchase a license, see
%
\begin{quote}
\url{http://software.intel.com/en-us/intel-mkl}
\end{quote}
%
For non-commercial development, see
%
\begin{quote}\small
\url{http://software.intel.com/non-commercial-software-development}
\end{quote}

\subsubsection{Installing and Compiling with MKL}

In order to use Intel's math kernel library (MKL) for \Cpp,
%
\begin{itemize}
\item  Download and extract a fresh copy of Stan.
\item  In your makefile change \code{CC=g++} to \code{CC=icc};  or
  you can do this by supplying the argument \code{CC=icc} to each
  call to make (aliases are good for this).
\item  Add the MKL path to the makefile;  for example
\begin{quote}
\begin{Verbatim}
MKLROOT = /apps/intel/2013/mkl)
\end{Verbatim}
\end{quote}
\item  Add the following to makefile's \code{CFLAGS}: 
\begin{quote}
\begin{Verbatim}
-I $(MKLROOT)/include and -DEIGEN_USE_MKL_ALL
\end{Verbatim}
\end{quote}
\item Link to your MKL library by adding to your makefile's
  \code{LDLIBS}.  The exact implementation will depend on your
  system. Use the MKL link line advisor for help. For example, you
  might add 
\begin{quote}
\begin{Verbatim}
-L$(MKLROOT)/lib/intel64 -lmkl_intel_lp64
   -lmkl_core -lmkl_sequential -lpthread -lm
\end{Verbatim}
\end{quote}
\item  Compile your models as usual, for example
\begin{quote}
\begin{Verbatim}
make src/models/speed/logistic/logistic
\end{Verbatim}
\end{quote}
\end{itemize}
%
{\it Note:} \ Make sure to do the above changes before compiling for the first
time - otherwise Stan will be compiled with \code{g++} and you won't see any
performance gains.  


\section{Required Software and Tools}

The only two absolute requirements for running \Stan are the
\Stan source code (and dependent libraries) and a \Cpp compiler.

\subsection{\Stan Source}

In order to compile \Stan models, the \Stan source code is required.
The latest version of \Stan can be downloaded from the following link.
%
\begin{quote}
\url{http://mc-stan.org/}
\end{quote}
%
The \Stan source code distribution includes \Stan's source code,
documentation, build tools, unit tests, demo models, documentation and
source for the required libraries Boost and Eigen, and the source for
an optional testing library, Google Test.

\subsubsection{Boost C++ Library Source}

\Stan's parser and some of its mathematical functions and 
template metaprogramming facilities are implemented with the Boost
\Cpp Library.  
%
\begin{itemize}
\item Home: http://www.boost.org/users/license.html
\item License: Boost Software License
\item Tested Version: 1.54.0
\end{itemize}
%
The Boost source code is distributed with \Stan.


\subsubsection{Eigen Matrix and Linear Algebra Library Source}

\Stan's matrix algebra depends on the Eigen \Cpp matrix and linear
algebra library.  
%
\begin{itemize}
\item Home: \url{http://eigen.tuxfamily.org}
\item License: Mozilla Public License, version 2.0
\item Tested Version: 3.2.0
\end{itemize}
%
The Eigen source code is distributed with \Stan.


\subsection{\Cpp Compiler}

Compiling \Stan models requires a \Cpp compiler.  \Stan has been
primarily developed with \clang and \gpp and no promises are made for
other compilers.  The full set of compilers for which \Stan has been
tested is
%
\begin{itemize}
%
\item \gpp
\\
Tested Versions: Mac 4.2.1, 4.6, Linux 4.4--4.7 (plus trunk 4.8, 4.9), Windows 4.6.3
\\
Home: \url{http://gcc.gnu.org/}
\\
License: GPL3+
%
\item \clang, Mac 2.9--3.1, Linux 2.9--3.1
\\
Home: \url{http://clang.llvm.org/}
\\
License: BSD
%
\item mingw-64, version 2.0 (Windows 7, cross-compiled from Debian Linux)
%
\item Intel \Cpp, Linux version 12.1.3
%
\end{itemize}
%

\subsubsection{\Cpp-11 Support} 

Stan 2.0 does not support \Cpp-11.  The remaining incompatibility with
the parser will be included soon after Stan 2.0 is released.  This
will include support for the latest versions of \code{g++} and
\code{clang++}.


\section{Optional Components for Developers}

\Stan is developed using the following set of tools.  The various
command examples in this manual have assumed they can be found on
the command path.  The makefile allows precise locations to be plugged
in. 

\subsection{GNU Make Build Tool}

\Stan automates the build, test, documentation, and deployment tasks
using scripts in the form of makefiles to run with GNU Make.
%
\begin{itemize}
\item Home: \url{http://www.gnu.org/software/make}
\item License: GPLv3+
\item Tested Versions: 3.81 (Mac OS X), 3.79 (Windows 7)
\end{itemize}
%


\subsection{Doxygen Documentation Generator}

\Stan's API documentation is generated using the Doxygen Tool.
%
\begin{itemize}
\item Home: \url{http://www.stack.nl/~dimitri/doxygen/index.html}
\item License: GPL2
\item Tested Version(s): Mac OS X 1.8.2, Windows 1.8.2
\end{itemize}


\subsection{Git Version Control System}

\Stan uses the Git version control system for its software, libraries,
and documentations.  Git is required to interact with the most recent
versions of code in the version control repository.
% 
\begin{itemize}
\item Home: \url{http://git-scm.com/}
\item License: GPL2
\item Tested Version(s): Mac version 1.7.8.4, Windows version 1.7.9
\end{itemize}


\subsubsection{Google Test C++ Testing Framework}

\Stan's unit testing is based on the Google's googletest \Cpp testing
framework.  
%
\begin{itemize}
\item
Home: \url{http://code.google.com/p/googletest/}
\item
License: BSD
\item
Tested Version(s): 1.6.0
\end{itemize}
%
The Google Test framework is distributed with \Stan.


\section{Tips for Mac OS X}

\subsection{Install Xcode}

Apple's Xcode contains both the \clang and \gpp compilers and make, all of the tools 
needed to work with \Stan as a user. 
The version of Xcode to install depends on the
version of Mac OS X.  

\subsubsection{Official Apple Xcode Distribution}

Xcode 4 may be downloaded for free for Mac OS X 10.7 (``Lion'') or
later directly from Apple:

\begin{quote}
Xcode 4: \url{https://developer.apple.com/xcode/}
\end{quote}

Once you've installed Xcode, you need to start it, then open
menu option \code{Xcode}, select \code{Preferences}, then click on the
\code{Downloads} icon and then click on the \code{Install} button next
to the option labeled ``Command Line Tools.''

At this point, you should have the make system \code{make} and the two
\Cpp compilers/linkers, \gpp and \clang, installed.  This is all you
need to run \Stan.  Xcode will also install the \code{git} version
control system at this point.

\subsubsection{Alternative, GCC-Only Installer}

A stripped down installer for just the GCC package, including the \Cpp
compilers \code{g++} and \code{clang++}, available for 
Mac OS X 10.6 (``Snow Leopard'') or later,
%
\begin{quote}
\url{https://github.com/kennethreitz/osx-gcc-installer/}
\end{quote}
%
The fill list of tools in this distribution is available at:
%
\begin{quote}
\url{http://www.opensource.apple.com/release/developer-tools-41/}
\end{quote}



\subsection{More Recent Compilers}

Alternative compilers to those distributed by Apple as part of Xcode
are available at the following locations.

\subsubsection{Homebrew}

One way to get pre-built binaries for Mac OS X is to use Homebrew,
which is available from the following link.
\begin{quote}
\url{http://mxcl.github.com/homebrew/}
\end{quote}

\subsubsection{MacPorts}

MacPorts hosts recent versions of compilers for the Macintosh.
%
\begin{quote}
\url{https://distfiles.macports.org/MacPorts/}
\end{quote}
%
After finding the appropriate \code{.dmg} file, clicking on it, then
double clicking on the resulting \code{.pkg} file, and clicking
through some more menus, the following will need to be entered from a
terminal window to install it.
%
\begin{quote}
\code{> sudo port install {\slshape gccVersion}}
\end{quote}
%
In this command, {\slshape gccVersion} is the name of a compiler
version, such as \code{g++=mp-4.6}, for version 4.6.  Errors may arise
during the install such as the following.
%
\begin{quote}\small\tt
  Error: Target org.macports.activate returned: Image error:
  /opt/local/include/gmp.h already exists and does not belong to a
  registered port.  Unable to activate port gmp. Use 'port -f activate
  gmp' to force the activation.
\end{quote}
%
This issue can be resolved by running the following command.
%
\begin{quote}
\code{> sudo port -f activate gmp}
\end{quote}
%


\subsection{Git Installer}

A standalone version of Git for Mac OS X is available from the
following site. 
%
\begin{quote}
\url{http://code.google.com/p/git-osx-installer/}
\end{quote}
%
Although (at the time of this writing) there were only versions listed
up to OS X version ``Snow Leopard,'' they work on ``Lion.''

\subsection{\LaTeX\ Typesetting Package}

\Stan uses the \LaTeX\ typesetting package for generating manuals,
talks, and other materials (Doxygen is used for API documentation; see
below).  The first step is to download the MacTeX \code{.mpkg} file
from the following URL [warning: the download is approximately 2GB and
the installation approximately 3.5GB].
%
\begin{quote}
\url{http://www.tug.org/mactex/2011/}
\end{quote}
%
Once it is downloaded, just click on the \code{.mpkg} file and then
follow the installer instructions.  The installer will add the command
to the \code{PATH} environment variable so that the \code{pdflatex}
used by \Stan is available from the command line.


\subsection{Lucida Console Font}

A free TrueType version of Lucida Console for the Mac is available
at the following URL.
%
\begin{quote}
\url{http://www.fontpalace.com/font-details/Lucida+Console/}
\end{quote}
%
Download the \code{.ttf} file, then click on it to install.  It
will then be available as a preference in the Mac terminal application.

\subsection{Doxygen API Documentation}

\Stan's API documentation is generated using the Doxygen tool.   This
tool is available from
%
\begin{quote}
\url{http://www.doxygen.org}
\end{quote}
%
Select the \code{Download} link from the second of the right-hand side
navigation bars, then select the binary distribution \code{.dmg} file
for Mac OS X.  Clicking on the \code{.dmg} file opens the finder with
a view of the unpacked Doxygen executable.  Just drag the Doxygen icon
into the Applications folder (or wherever you want to keep it).  Then
add the path to the Doxygen executable, 
%
\begin{quote}
\url{/Applications/Doxygen.app/Contents/Resources/doxygen}
\end{quote}
%
to the system \code{PATH} environment variable.  You can do add to the
\code{PATH} environment by adding this line to the end of the
top-level \Verb|~/.profile| file.  
%
\begin{quote}
\begin{Verbatim}[fontsize=\small]
export PATH=/Applications/Doxygen.app/Contents/Resources:$PATH
\end{Verbatim}
\end{quote}
%
The next shell started will then be able to find the \code{doxygen} command.


\section{Tips for Windows}

\subsection{Install Rtools}

The easiest way to get a complete \Cpp build environment on Windows is
to install the most recent version of Rtools.  

The latest version verified to work with \Stan is Rtools 2.15.  Rtools
2.15 includes the \gpp 4.6.3 (pre-release) compiler and many other
useful command line tools including many Unix commands, such as the
following.
%
\begin{quote}
\tt basename, cat, cmp, comm, cp, cut, date,
diff, du, echo, expr, gzip, ls, make, makeinfo, mkdir, mv, rm, rsync,
sed, sh, sort, tar, texindex, touch, uniq
\end{quote}

Rtools can be downloaded from the following location.
%
\begin{quote}
  \url{http://cran.r-project.org/bin/windows/Rtools/}
\end{quote}
%
Install it using the Windows installer.  Allow it to edit the
\code{PATH} environment variable so that commands are available from
the command tool.

To verify the installation was successful, open a command window by
selecting the following menu items.
%
\begin{quote}
  \code{Start} 
  $\rightarrow$ Accessories 
  $\rightarrow$ Command Prompt
\end{quote}
%
To verify that \gpp is installed, use the following command.
%
\begin{quote}
  \Verb|> g++ -v|
\end{quote}
%
This should report version information for \gpp.  Next, verify that
\code{make} is installed with the following command.
%
\begin{quote}
  \Verb|> make -v|
\end{quote}
%
This should print version information for make.

\subsection{GNU Make 3.81 or Higher for Tests}\label{windows-make-install.section}

Although the version of \code{make} distributed with \code{RTools}
suffices to run Stan, in order to run the unit tests (see \refsection{testing-stan}),
a version of GNU \code{make} version 3.81 or higher is required.  To
install such a version of \code{make}:
%
\begin{itemize}
\item Install \code{Rtools} according to instructions in the previous section.
\item Download and install make 3.81 (or higher).  
  \begin{itemize}
    \item we have tested the version installed through the
      \code{Setup} link at \\
      \url{http://gnuwin32.sourceforge.net/packages/make.htm}
  \end{itemize}
\item Edit the \code{PATH} environment variable so the installation
  location of \code{make} appears before \code{RTools}.
\item Remove or move Windows' \code{find.exe} from
  \Verb|C:\Windows system32|.  For example, move it to the top-level
  \code{C:} directory.%
\end{itemize}
%
The reason for the last step is a bug in this version of \code{make}. 
Even though it should pick out \code{find.exe} from \code{Rtools}, 
it picks out \Verb|C:\Windows\system32| first.

\subsection{Install Git}

There are a number of Git clients for Windows that will work.  The
official Git installer for Windows can be found at the following
location.
%
\begin{quote}
\url{http://code.google.com/p/msysgit/downloads}
\end{quote}
%
Select the latest full installer and install it. 
