\chapter{Installation and Compatibility}\label{install.appendix}

\noindent
This appendix describes the hardware and software required to run
\CmdStan.  The software required includes \CmdStan and its libraries,
as well as a contemporary \Cpp compiler.  \CmdStan requires hardware
powerful enough to build and execute the models.  Ideally, that will
be a 64-bit computer with at least 4GB of memory and multiple
processor cores.

\section{Operating System}

\CmdStan is written in portable \Cpp with {\Cpp}11 and {\Cpp}14 features, as are the
libraries on which it depends.  Therefore, \CmdStan should run on any machine
for which a suitable \Cpp compiler supporting C++1y or C++14 features is available. In practice, \CmdStan,
like the Boost and Eigen libraries on which it depends, is very hard on the compiler and linker.

\CmdStan has been tested on the following operating systems.
%
\begin{itemize}
\item Linux (Debian, Ubuntu)
\item Mac OS X (from 10.6 ``Snow Leopard''  through 10.14 ``Mojave'')
\item Windows (7, 8, 10).
\end{itemize}
%
\CmdStan should work on other versions of these operating systems if
compatible \Cpp compilers can be found.  The plan is to keep up with
new versions of these operating systems and gradually phase out
testing on older versions.

\section{Required Software and Tools}

The only two absolute requirements for running \CmdStan are the
\CmdStan source code (and dependent libraries) and a \Cpp compiler.

\subsection{\CmdStan Source}

In order to compile Stan program, the \CmdStan source code is
required.  The \CmdStan source code distribution includes \CmdStan's
source code, Stan's source code, documentation, build scripts, unit
tests, documentation and source for the required libraries, Stan Math
Library, Boost, and Eigen, and the source for an optional testing
library, Google Test.

\subsubsection{CmdStan: Stable Releases}

The latest release version of \CmdStan can be downloaded
from the CmdStan home page:
%
\begin{quote}
\url{http://mc-stan.org/cmdstan.html}
\end{quote}
%

\subsubsection{CmdStan: Development Source Control}

The source code repository is hosted by GitHub, and contains the
latest versions of CmdStan (and Stan) underdevelopment.  See:
%
\begin{quote}
http://mc-stan.org/source-repos.html
\end{quote}

\subsubsection{Stan Library Source}

The source code for Stan's parse and implementation of inference
algorithms are in the Stan library.
%
\begin{itemize}
\item Home: \url{http://mc-stan.org/stan.html}
\item License: BSD
\item Tested Version: 2.20.0
\end{itemize}
%
The Stan source code is distributed with \CmdStan.

\subsubsection{Stan Math Library Source}

Stan's mathematical functions and inference algorithm rely on the
reverse-mode automatic differentiation implemented in the Stan Math
Library.
%
\begin{itemize}
\item Home: \url{http://mc-stan.org}
\item License: BSD
\item Tested Version: 2.20.0
\end{itemize}
%
The Stan Math Library source code is distributed with \CmdStan

\subsubsection{Boost C++ Library Source}

Stan's parser and some of its mathematical functions and template
metaprogramming facilities are implemented with the Boost \Cpp
Library.
%
\begin{itemize}
\item Home: \url{http://www.boost.org/users/license.html}
\item License: Boost Software License
\item Tested Version: 1.69.0
\end{itemize}
%
The Boost source code is distributed with \CmdStan.


\subsubsection{Eigen Matrix and Linear Algebra Library Source}

Stan's matrix algebra depends on the Eigen \Cpp matrix and linear
algebra library.
%
\begin{itemize}
\item Home: \url{http://eigen.tuxfamily.org}
\item License: Mozilla Public License, version 2.0
\item Tested Version: 3.3.3
\end{itemize}
%
The Eigen source code is distributed with \CmdStan.


\subsection{\Cpp Compiler}

Compiling \CmdStan programs requires a \Cpp compiler.  \CmdStan has been
primarily developed with \clang and \gpp and no promises are made for
other compilers.  The full set of compilers for which \CmdStan has been
tested is listed here:
\url {https://github.com/stan-dev/stan/wiki/Supported-C---Compilers-and-Language-Features}

\section{Step-by-Step Windows Install Instructions}\label{install-windows.appendix}

\CmdStan has been tested on Windows XP, Windows 7, and Windows 8.

\CmdStan also runs under Cygwin, which provides a unix-like shell on top
of Windows.  Instructions for Cygwin installation are provided below
in their own subsection.


\subsection{Windows Tips}


\subsubsection{Opening a Command Shell}

To open a Windows command shell, first open the \code{Start Menu}
(usually in the lower left of the screen), select option
\code{All Programs}, then option \code{Accessories}, then
program \code{Command Prompt}.

Alternatively, enter \code{[Windows+r]} (both keys together on the
keyboard), and enter \code{cmd} into the text field that pops up in
the Run window, then press \code{[Return]} on the keyboard to run.

\subsubsection{32-bit Builds}

\CmdStan defaults to a 64-bit build. On a 32-bit operating system,
include \verb|BIT=32| in \code{make/local}. See
\refappendix{make-options} for more details.

\subsection{Rtools C++ Development Environment}

The simplest way to install a full \Cpp build environment that will
work for \CmdStan is to use the Rtools package designed for R
developers on Windows (even if you don't plan to use R).

First, download the latest \emph{frozen} (i.e., stable) version of
Rtools from the Rtools home page, using
%
\begin{quote}
\url{http://cran.r-project.org/bin/windows/Rtools/}
\end{quote}
%
Next, double click on the downloaded file to open the Rtools
install wizard, then proceed through its options.
\begin{itemize}
\item \emph{Language}: select language, click \code{Next},
\item \emph{Welcome}: click \code{Next},
\item \emph{Information}: click \code{Next},
\item \emph{Setup Location}: accept default (\Verb|c:\Rtools|), click \code{Next},
\item \emph{Select Components}: select default, \code{Package
   Authoring}, click \code{Next},
\item \emph{Select Additional Tasks}: check \code{Edit Path} and \code{Save
 Version in Registry}, click \code{Next},
\item \emph{System Path Report}: ensure that that the paths to \code{c:\textbackslash{}Rtools\textbackslash{}bin} and \code{c:\textbackslash{}Rtools\textbackslash{}gcc-4.6.3\textbackslash{}bin} are listed at the beginning of the path and click \code{Next},
\item \emph{Ready to Install}: click \code{Next}, wait for the
  install to complete, then
\item \emph{Finish}: click \code{Finish}.
\item \emph{Confirm Path}: After the install has completed, open a command prompt and type \code{PATH} to ensure that the new path is activated and the Rtools folders are in the system path.
\end{itemize}

\subsubsection{Checking the Path}

Make sure that \Verb|c:\Rtools\bin| has been added to your \Verb|PATH|
environment variable, and then open another command window.  You
should be able to follow the last step, \emph{Comfirm Path}, above.

Note: if you see an error like ``\code{FIND: Parameter format
  not correct},'' please check to see that \Verb|c:\Rtools\bin| is
listed at the beginning of the path.

\subsection{Verifying Tools}

To verify that \gpp is installed, use the following command.
%
\begin{quote}
  \Verb|> g++ -v|
\end{quote}
%
This should report version information for \gpp.  Next, verify that
\code{make} is installed with the following command.
%
\begin{quote}
  \Verb|> make -v|
\end{quote}
%
This should print version information for make.

\subsection{Downloading and Unpacking \CmdStan}

The \CmdStan source code distribution is named
\code{cmdstan-\cmdstanversion.tar.gz}; the versions here are major
version 2, minor version 7, and patch level 0.  Download the latest
\CmdStan source tarball from the \CmdStan downloads page,
%
\begin{quote}
\url{https://github.com/stan-dev/cmdstan/releases}
\end{quote}
%
to any non-temporary folder.  (If in doubt, select \code{My Documents}
on Windows XP or \code{Documents} on Windows 7.)

Change to the download directory (aka folder) using one of the
following commands, replacing \code{<username>} with
a Windows user name.
%
\begin{itemize}
\item \emph{Windows XP}: \ From the default starting directory, use
the following commands (quotes and all):
\begin{quote}
\begin{Verbatim}[fontshape=sl,fontsize=\small]
> cd "My Documents"
\end{Verbatim}
\end{quote}
%
The full path (including quotes) will work from anywhere,
\begin{quote}
\begin{Verbatim}[fontshape=sl,fontsize=\small]
> cd "c:\Documents and Settings\<username>\My Documents"
\end{Verbatim}
\end{quote}
\item \emph{Windows 7}:  \  From the default starting directory, use
\begin{quote}
\begin{Verbatim}[fontshape=sl,fontsize=\small]
> cd Documents
\end{Verbatim}
\end{quote}
or use the full path, including quotes, from anywhere,
\begin{quote}
\begin{Verbatim}[fontshape=sl,fontsize=\small]
> cd "c:\Users\<username>\Documents"
\end{Verbatim}
\end{quote}
\end{itemize}
%
To verify that the downloaded \CmdStan \code{.tar.gz} file is there,
list the directory contents using:
%
\begin{quote}
\begin{Verbatim}[fontshape=sl,fontsize=\small]
> dir
\end{Verbatim}
\end{quote}

Finally, unpack the distribution using the \code{tar} command (which
is installed as part of Rtools).
%
\begin{quote}
\begin{Verbatim}[fontshape=sl,fontsize=\small]
> tar --no-same-owner -xzf cmdstan-2.20.0.tar.gz
\end{Verbatim}
\end{quote}
%
The \code{--no-same-owner} flag is not strictly necessary,
but it removes a bunch of irrelevant warnings.


\subsection{64-bit Cygwin Install Instructions}

\CmdStan can be run under Cygwin, the Unix look-and-feel environment
for Windows.  Cygwin must have recent versions of \code{make} and
\code{g++} (part of gcc) installed.  Within a Cygwin shell, \CmdStan
will behave as under other Unixes. Follow the directions in
\refappendix{install-linux}.



\section{Step-by-Step Mac Install Instructions}\label{install-mac.appendix}

This section provides step-by-step install instructions for the Mac.
\CmdStan has been tested on Mac OS X versions Mavericks, Snow Leopard,
Lion, and Mountain Lion.

\subsection{Install Xcode C++ Development Environment}

The easiest (but not the only) way to install a \Cpp development
environment on a Mac is to use Apple's Xcode development environment.

From the Xcode home page,
%
\begin{quote}
\url{https://developer.apple.com/xcode/}
\end{quote}
%
click \code{View in Mac App Store}.

From the App Store, click \code{Install}, enter an Apple ID,
and wait for Xcode to finish installing.

Open the Xcode application, click top-level menu
\code{Preferences}, click top-row button \code{Downloads},
click button for \code{Components}, click on the \code{Install}
button to the right of the \code{Command Line Tools} entry, then
wait for it to finish installing.

Click the top-level menu item \code{Xcode}, then click item
\code{Quit Xcode} to quit.

To test, open the Terminal application and enter

\begin{quote}
\begin{Verbatim}[fontshape=sl,fontsize=\small]
> make --version
> g++ --version
\end{Verbatim}
\end{quote}
%
Verify that \code{make} is at version 3.81 or later and \code{g++}
is at 4.9.3 or later.


\subsection{Download and Unpack \CmdStan Source}

Download the most recent version of \code{cmdstan-\cmdstanversion.tar.gz} from
the \CmdStan downloads list,
%
\begin{quote}
\url{https://github.com/stan-dev/cmdstan/releases}
\end{quote}

Open the folder containing the download in the Finder
(typically, the user's top-level \code{Downloads} folder).

If the Mac OS has not automatically unpacked the \code{.tar.gz}
file into file \code{cmdstan-\cmdstanversion.tar},
double-click the \code{.tar.gz} file to unpack.

Double click on the \code{.tar} file to unarchive
directory \code{cmdstan-\cmdstanversion}.

Move the resulting directory to a location where it will not be
deleted, henceforth called \code{<cmdstan-home>}.


\section{Step-by-Step Linux Install Instructions}\label{install-linux.appendix}

\CmdStan has been tested on various Linux installations, including
Ubuntu, Debian, and Red Hat.

\subsection{Installing C++ Development Tools}

On Linux, \Cpp compilers and \code{make} are often installed by
default.

To see if the \code{g++} compiler and \code{make} build system
are already installed, use the commands
%
\begin{quote}
\begin{Verbatim}[fontshape=sl,fontsize=\small]
> g++ --version
> make --version
\end{Verbatim}
\end{quote}
%
If these are at least at \code{g++} version 4.9.3 or later and
\code{make} version 3.81 or later, no additional installations are
necessary.  It may still be desirable to update the \Cpp compiler
\code{g++}, because later versions are faster.

To install the latest version of these 
tools (or upgrade an older version), use the following commands or their equivalent for your distribution: 
%
\begin{quote}
\begin{Verbatim}[fontshape=sl,fontsize=\small]
> sudo apt install g++
> sudo apt install make
\end{Verbatim}
\end{quote}
%
A password will likely be required by the superuser command \code{sudo}.


\subsection{Downloading and Unpacking \CmdStan Source}

Download the most recent stable version of \CmdStan,
\code{cmdstan-\cmdstanversion.tar.gz}, from the \CmdStan downloads page,
%
\begin{quote}
\url{https://github.com/stan-dev/cmdstan/releases}
\end{quote}
%
to the directory where Stan will reside.

In a command shell, change directories to where the tarball was
downloaded, say \code{<download-dir>}, with
%
\begin{quote}
\begin{Verbatim}[fontshape=sl,fontsize=\small]
> cd <download-dir>
\end{Verbatim}
\end{quote}
%
where \code{<download-dir>} is replaced with the actual path to the directory.

Then, unpack the distribution into the subdirectory
\begin{quote}
\nolinkurl{<download-dir>/cmdstan-\cmdstanversion}
\end{quote}
%
with
%
\begin{quote}
\begin{Verbatim}[fontshape=sl,fontsize=\small]
> tar -xzf cmdstan-2.20.0.tar.gz
\end{Verbatim}
\end{quote}


\section{{\tt make} Options}\label{make-options.appendix}

\subsection{Setting a Variable}
\CmdStan uses \code{make} for building the \CmdStan tools and
compiling Stan programs as executables. Users can customize how the
tools are built by creating and editing \code{make/local} inside their
\code{<cmdstan-home>} directory.

Customization is done by setting variables or appending to them. For
each variable that needs to be set, write \Verb|<variable>=<value>| on
its own line inside \code{make/local}. To append to existing
variables, write \Verb|<variable>+=<value>|.

\subsection{Customizing {\tt make} Options}

\subsubsection{Compiler Settings}

The defaults should work for most Windows, Linux, and Mac setups. The most common options to customize the build are:
%
\begin{itemize}
  \item \Verb|CXX|. The compiler used to build \CmdStan and the Stan
    executables. The default is \Cpp compiler on the system.
  \item \Verb|O|. The optimization level for the compiler. The default
    is \Verb|O=3|. Both \gpp and \clang recognize \Verb|0,1,2,3|, and
    \Verb|s|.
  \item \Verb|O_STANC|. The optimization level for building the Stan compiler.
    The default is \Verb|O=3| on Windows and \Verb|O=0| for other operating systems.
  \item \Verb|CXXFLAGS_OS|. Any additional compiler flags that would be useful to set.
  \item \Verb|INC_FIRST|. Any header files to include before the rest of the Stan includes.
\end{itemize}

\subsubsection{{\tt stanc} Options}

There are two options for controlling \stanc:
%
\begin{itemize}
\item \Verb|STANCFLAGS|. These flags are passed to \stanc.  To allow
  functions in Stan programs that are declared, but undefined, use
  \Verb|STANCFLAGS = --allow_undefined| and update \Verb|USER_HEADER|
  appropriately.
\item \Verb|USER_HEADER|. If \Verb|--allow_undefined| is passed to
  \stanc, the value of this variable is the additional include that is
  passed to the \Cpp linker when compiling the program. The file in
  \Verb|USER_HEADER| must have a definition of the function if the function
  is used and not defined in the Stan program. This defaults to
  \Verb|user_header.hpp| in the directory of the Stan program being compiled.
\end{itemize}
%

\subsubsection{Changing Library Locations}

The next set of variables allow for easy replacement of dependent
libraries. \CmdStan depends on Stan, Boost, Eigen, and
CVODES. Although \CmdStan is bundled with a particular verison of
Stan, other versions can be used. The same is true with Boost, Eigen,
and CVODES.

The Stan developers typically replace the tagged version of Stan in
\code{<cmdstan-home>/stan} with an updated version and do not set
these variables. That said, it is easy to point \CmdStan to other
versions of these libraries in other directories.
%
\begin{itemize}
  \item \Verb|STAN|. The location of the Stan source
    directory. The default is \Verb|STAN=stan_2.20.0/|. Note: the
    trailing forward slash is necessary.
  \item \Verb|MATH|. The location of the Stan Math Library. The
    default is \Verb|MATH=stan/l../stan_math_2.20.0/|.
  \item \Verb|BOOST|. The location of the Boost library. If
    \Verb|BOOST| is not explicitly set, this will default to the
    \Verb|lib/boost_1.66.0/| folder within \Verb|MATH|.
  \item \Verb|EIGEN|. The location of the Eigen library. If
    \Verb|EIGEN| is not explicitly set, this will default to the
    \Verb|lib/eigen_3.3.3/| folder within \Verb|MATH|.
  \item \Verb|SUNDIALS|. The location of the SUNDIALS library. If
    \Verb|SUNDIALS| is not explicity set, this will default to the
    \Verb|lib/sundials_3.1.0/| folder with \Verb|MATH|.
\end{itemize}
%


\subsection{Rebuilding \CmdStan}

When compiler flags are changed, \CmdStan needs to be rebuilt in order
for the changes to take place. The easiest way to do this is to type:
%
\begin{quote}
  \begin{Verbatim}[fontshape=sl]
> make clean-all
  \end{Verbatim}
\end{quote}
%
Then rebuild \CmdStan and the Stan programs of interest.


\section{MKL Compiler Instructions}

\subsection{Getting the MKL}

To purchase a license, see
%
\begin{quote}
\url{http://software.intel.com/en-us/intel-mkl}
\end{quote}
%
For non-commercial development, see
%
\begin{quote}\small
\url{http://software.intel.com/non-commercial-software-development}
\end{quote}

\subsection{Compiling with MKL}

In order to use Intel's math kernel library (MKL) for \Cpp, a few
lines need to be added to \code{make/local}.
\begin{itemize}
  \item \Verb|CXX = icc|. Set the compiler to icc.
  \item \Verb|MKLROOT = /apps/intel/2013/mkl|. Create a new variable
    with the location of the MKL path.
  \item
    \Verb|CXXFLAGS += -I $(MKLROOT)/include -DEIGEN_USE_MKL_ALL|. This
    tells Eigen to use MKL and where to find the necessary header
    files.
  \item \Verb|LDLIBS += -L$(MKLROOT)/lib/intel64 -lmkl_intel_lp64| and
    \Verb|LDLIBS += -lmkl_core -lmkl_sequential -lpthread -lm|. This
    links in the MKL library for the \CmdStan executables. The exact
    implementation will depend on the particular system. Use the MKL
    link line advisor for help.
\end{itemize}
%
{\it Note:} \ Make sure to do the above changes before compiling for the first
time - otherwise Stan will be compiled with \code{g++} and you won't see any
performance gains.

\subsection{Additional compiler options}

By default, Intel's compiler trades off accuracy for speed. This,
unfortunately, isn't ideal behavior for the inference algorithms and
may cause divergent transitions to go {\em unreported}. Add these
flags to \code{make/local} to force more accurate floating point
computations which ensure that divergent transitions are reported:
\begin{itemize}
  \item \Verb|CXXFLAGS += -fp-model precise -fp-model source|
\end{itemize}


\section{Optional Parallelization Support}

\CmdStan has optional support for within-chain parallelization using
threading, OpenCL, or MPI (Message Passing Interface).

\subsection{Threading and MPI}
%
\begin{description}
  \item[Threads] can utilize multiple cores on a single
    machine. Please refer to
    \url{https://github.com/stan-dev/math/wiki/Threading-Support} for
    instructions on how to setup threading.
  \item[MPI] is typically used on large computing clusters. The MPI
    standard allows to link together a large number of processes which
    can run locally on one machine or across many
    different machines. Please refer to
    \url{https://github.com/stan-dev/math/wiki/MPI-Parallelism} for
    more details.
\end{description}
%
The within-chain parallelization for threading and MPI is used in the Stan
language to parallelize the evaluation of the \Verb|map_rect| command. Note that
MPI parallelism will take precendence in case both features are
enabled.

\subsection{OpenCL}

OpenCL in \CmdStan enables Cholesky decompositions to utilize a GPU for large parallel computations. In order to use OpenCL, a few lines need to be added to \code{make/local}.
\begin{itemize}
  \item \Verb|STAN_OPENCL=true| Enable the use of OpenCL.
  \item \Verb|OPENCL_PLATFORM_ID=<int>| Set which OpenCL platform to use.
  \item \Verb|OPENCL_DEVICE_ID=<int>| Set which OpenCL device on the selected platform to use.
\end{itemize}

Windows users need to additionally specify the following two flags depending on the GPU you are running OpenCL on.

\subsubsection{NVIDIA GPU}
\begin{itemize}
  \item \Verb|LDFLAGS_OPENCL= -L"\$(CUDA_PATH)\lib\x64" -lOpenCL|
\end{itemize}

\subsubsection{AMD GPU}
\begin{itemize}
  \item \Verb|LDFLAGS_OPENCL= -L"\$(AMDAPPSDKROOT)lib\x86_64" -lOpenCL|
\end{itemize}

If you are unsure if your devices support OpenCL, we suggest running the \code{clinfo} application that lists all available OpenCL supported devices. Please refer to \url{https://github.com/stan-dev/math/wiki/OpenCL-GPU-Routines} for additional setup instructions. Currently, OpenCL will only pass matrices over to the GPU if their dimensions are larger than $1250\times 1250$.

\section{Optional Components for Developers}

\CmdStan is developed using the following set of tools.  The various
command examples in this manual have assumed they can be found on
the command path.  The makefile allows precise locations to be plugged
in.

\subsection{GNU Make Build Tool}

\CmdStan automates the build, test, documentation, and deployment tasks
using scripts in the form of makefiles to run with GNU Make.
%
\begin{itemize}
\item Home: \url{http://www.gnu.org/software/make}
\item License: GPLv3+
\item Tested Versions: 3.81 (Mac OS X), 3.79 (Windows 7)
\end{itemize}
%


\subsection{Doxygen Documentation Generator}

\CmdStan's API documentation is generated using the Doxygen Tool.
%
\begin{itemize}
\item Home: \url{http://www.stack.nl/~dimitri/doxygen/index.html}
\item License: GPL2
\item Tested Version(s): Mac OS X 1.8.2, Windows 1.8.2
\end{itemize}


\subsection{Git Version Control System}

\CmdStan uses the Git version control system for its software, libraries,
and documentations.  Git is required to interact with the most recent
versions of code in the version control repository.
%
\begin{itemize}
\item Home: \url{http://git-scm.com/}
\item License: GPL2
\item Tested Version(s): Mac versions 1.8.2.3 and 1.7.8.4; Windows version 1.7.9
\end{itemize}


\subsubsection{Google Test C++ Testing Framework}

\CmdStan's unit testing is based on the Google's googletest \Cpp testing
framework.
%
\begin{itemize}
\item
Home: \url{http://code.google.com/p/googletest/}
\item
License: BSD
\item
Tested Version(s): 1.8.1
\end{itemize}
%
The Google Test framework is distributed with Stan.


\section{Tips for Mac OS X}

\subsubsection{Finding and Opening Mac Applications and Files}

To open an application, use \code{[Command-Space]} (press both keys at
once on the keyboard) to open Spotlight, enter the application's name
in the text field, then click on the application in the pop-up menu or
\code{[Return]} if the right file or application is highlighted.

Spotlight can be used in the same way to find files or folders,
such as the default \code{Downloads} folder for web downloads.

\subsubsection{Open a Terminal for Shell Commands}

To run shell commands, open the built-in Terminal application (see the
previous subsection for details on how to find and open applications).

\subsection{Install Xcode}

Apple's Xcode contains both the \clang and \gpp compilers and make,
all of the tools needed to work with \CmdStan as a user.  The version
of Xcode to install depends on the version of Mac OS X.

\subsubsection{Alternative, GCC-Only Installer}

A stripped down installer for just the GCC package, including the \Cpp
compilers \code{g++} and \code{clang++}, available for Mac OS X 10.6
(``Snow Leopard'') or later,
%
\begin{quote}
\url{https://github.com/kennethreitz/osx-gcc-installer/}
\end{quote}
%
The fill list of tools in this distribution is available at:
%
\begin{quote}
\url{http://www.opensource.apple.com/release/developer-tools-41/}
\end{quote}


\subsection{More Recent Compilers}

Alternative compilers to those distributed by Apple as part of Xcode
are available at the following locations.

\subsubsection{Homebrew}

One way to get pre-built binaries for Mac OS X is to use Homebrew,
which is available from the following link.
\begin{quote}
\url{http://mxcl.github.com/homebrew/}
\end{quote}

\subsubsection{MacPorts}

MacPorts hosts recent versions of compilers for the Macintosh.
%
\begin{quote}
\url{https://distfiles.macports.org/MacPorts/}
\end{quote}
%
After finding the appropriate \code{.dmg} file, clicking on it, then
double clicking on the resulting \code{.pkg} file, and clicking
through some more menus, the following will need to be entered from a
terminal window to install it.
%
\begin{quote}
\code{> sudo port install {\slshape gccVersion}}
\end{quote}
%
In this command, {\slshape gccVersion} is the name of a compiler
version, such as \code{g++-mp-4.6}, for version 4.6.  Errors may arise
during the install such as the following.
%
\begin{quote}\small\tt
  Error: Target org.macports.activate returned: Image error:
  /opt/local/include/gmp.h already exists and does not belong to a
  registered port.  Unable to activate port gmp. Use 'port -f activate
  gmp' to force the activation.
\end{quote}
%
This issue can be resolved by running the following command.
%
\begin{quote}
\code{> sudo port -f activate gmp}
\end{quote}
%

\subsection{\LaTeX\ Typesetting Package}

\CmdStan uses the \LaTeX\ typesetting package for generating manuals,
talks, and other materials (Doxygen is used for API documentation; see
below).  The first step is to download the MacTeX \code{.mpkg} file
from the following URL [warning: the download is approximately 2GB and
the installation approximately 3.5GB].
%
\begin{quote}
\url{http://www.tug.org/mactex/2011/}
\end{quote}
%
Once it is downloaded, just click on the \code{.mpkg} file and then
follow the installer instructions.  The installer will add the command
to the \code{PATH} environment variable so that the \code{pdflatex}
used by Stan is available from the command line.


%% \subsection{Lucida Console Font}

%% A free TrueType version of Lucida Console for the Mac is available
%% at the following URL.
%% %
%% \begin{quote}
%% \url{http://www.fontpalace.com/font-details/Lucida+Console/}
%% \end{quote}
%% %
%% Download the \code{.ttf} file, then click on it to install.  It
%% will then be available as a preference in the Mac terminal application.
